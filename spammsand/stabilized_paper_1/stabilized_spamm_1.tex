%% LyX 2.1.1 created this file.  For more info, see http://www.lyx.org/.
%% Do not edit unless you really know what you are doing.
%\documentclass[letterpaper,twocol,amsmath,aps,jcp,preprintnumbers,groupaddress,nofootinbib,tightenlines]{revtex4-1}
%\usepackage[latin9]{inputenc}
%\usepackage{textcomp}

\documentclass[letterpaper,twocolumn,amsmath,amsfont,amssymb,english,aps,jcp,preprintnumbers,groupaddress,nofootinbib,tightenlines]{revtex4}

\usepackage{graphicx}


%\documentclass[aps,prb,letterpaper,twocolumn,nofootinbib,showkeys]{revtex4-1}
%\documentclass[aps,amssymb,prl,letterpaper,twocolumn,nofootinbib,showkeys]{revtex4-1}

%\usepackage[backend=bibtex]{biblatex}

%    backend=biber,
%    style=authoryear,
%    natbib=true,
%    sortlocale=en_US,
%    url=false,
%    doi=true,f
%    eprint=false
%]{biblatex}
%\usepackage{hyperref}


\newcommand{\mat}[1]{\boldsymbol{#1}}
\newcommand{\matT}[1]{\boldsymbol{#1}^\dagger}
\newcommand{\ot}{ {\scriptstyle \otimes}_{ \tau } }

%%\hypersetup{pdftitle={FreeON Project Report 1}}
%\hypersetup{pdfauthor={Matt Challacombe and Nicolas Bock}}
%\hypersetup{pdfsubject={A SpAMM Stabilized Newton Schulz Preconditioner: Fighting Error with Error}}

%\bibstyle{aipnum4-1}

\begin{document}

\title{On Stability of Newton Schulz Iterations in an Approximate Algebra}

\author{Matt Challacombe}
\email{matt.challacombe@freeon.org}
\homepage{http://www.freeon.org}
\affiliation{Theoretical Division, Los Alamos National Laboratory}

\author{Nicolas Bock}
\email{nicolasbock@freeon.org}
\homepage{http://www.freeon.org}
\affiliation{Theoretical Division, Los Alamos National Laboratory}

%\begin{abstract}
%Forward look
%\end{abstract}

\maketitle
\section{Introduction}

In many disciplines, finite correlations coorespond to matrices with decay properties.  Matrix decay involves an approximate (perhaps bounded) inverse 
relationship between matrix elements and a cooresponding distance;  this may be a simple inverse exponential relationship between elements and the
Cartesian distance between support functions, or it may involve a statistical distance, e.g. between strings.  In electronic structure, 
correlations manifest in decay properties of the matrix 
sign function, as projector of the effective Hamiltonian.  More broadly, matrix decay properties may coorespond to learned correlations in a 
generalized, non-orthogonal metric, obtained perhaps through first order optimization involving the celebrated PLSEV line search approach to 
semi-definate programming based on the matrix sign function.  More broadly still, problems with local, non-orothogonal support
are often solved with congruential transformations based on the matrix inverse square root or inverse factor; these  transformations 
correlate the non-orthogonal support in a representation independent form, {\em eg.} of the generalized eigenproblem.    

These problems are related through Higham's identity, connecting the matrix sign function with the inverse square root:
\begin{equation}
.
\end{equation}
Computation of these matrix functions may encounter ill-conditioning, cooresponding to extended (quasi-degenerate) correlations and near 
complete non-orthogonal support and associated slow rates of decay.  For extremely slow decay, maybe even oscillatory, 
low order algebraic decay, methods that compresion....     For fast decay, 

correlation and the support  


Also, matrices with decay arise from the application of .  Generally, ill-conditining is associated with slower decay, 

Decay principles, often very sparse but very ill-conditioned problems.

\subsection{Incomplete/Inexact Schemes}

Incompleteness -> sparse approximations dense problems, uses conventional sparse infrastructure, second order errors in matrix multiplication.  
Often adhoc.

$\tilde{\mat{a}} = \mat{a}+ \delta \mat{a}$

$\tilde{\mat{a}} \cdot \tilde{\mat{b}} = \mat{a}\cdot\mat{b}+ \delta \mat{a} \cdot \mat{b} +  \mat{a} \cdot \delta \mat{b}
+  \delta \mat{a} \cdot \delta \mat{b}$

The variations do not express in the overal context of the product.  Because the error in the incomplete case is additive, the

 For example, 
$\mat{a}$ may be small, but $\delta \mat{a} \cdot \mat{b}$  large, leading extra work.  Also, 
once a truncation error is commited, it is encounted in all subsiquent steps; it becomes difficult to impossilbe to manage error flows of differing 
magnitude in complex maps. 


\subsection{Retaining the Eigenspace}
Gradients lack convergence properties
Iteration without orig drives away from basis
NS has both.  Difference between scalar iteration, Higham page 92.


\subsection{Approximate Algebra as {\bf \em N}-Body Problem}


SpAMM is the recursive Cauchy-Schwarz occlusion product $\ot$ on matrix quadtrees

\begin{equation}
\mat{a}^i = \begin{bmatrix} \mat{a}^{i+1}_{00} & \mat{a}^{i+1}_{01} \\ \mat{a}^{i+1}_{10} & \mat{a}^{i+1}_{11} \end{bmatrix}
\end{equation}

\begin{widetext}
\begin{equation}
\mat{a}^{i} \ot \mat{b}^{i} = 
\left\{
        \begin{array}{ll}
                 \emptyset \quad \tt{if}\quad \lVert \mat{a}^i \rVert \lVert \mat{b}^i \rVert < \tau \\[0.2cm]
                 \mat{a} ^i \cdot \mat{b}^i \quad  \tt{if}(i=\tt{leaf}) \\[0.2cm]
\begin{bmatrix} \mat{a}^{i+1}_{00} \ot \mat{b}^{i+1}_{00} +\mat{a}^{i+1}_{01} \ot \mat{b}^{i+1}_{10} \; , \; &
                \mat{a}^{i+1}_{00} \ot \mat{b}^{i+1}_{01} +\mat{a}^{i+1}_{01} \ot \mat{b}^{i+1}_{11}  \\[0.2cm] 
                \mat{a}^{i+1}_{00} \ot \mat{b}^{i+1}_{01} +\mat{a}^{i+1}_{01} \ot \mat{b}^{i+1}_{11} \; , \; & 
                \mat{a}^{i+1}_{00} \ot \mat{b}^{i+1}_{01} +\mat{a}^{i+1}_{01} \ot \mat{b}^{i+1}_{11}   
\end{bmatrix}  \quad \tt{else}
                \end{array}
              \right.
\end{equation}
\end{widetext}




database orientation, Cauchy sch
Approximate Algebra, SpAMM Cauchy Schwarz occlusion, n-body approach to numerical linear algebra, first order errors in matrix multiplication.
Based on Cauchy Schwarz inequality.



% error accumulation may be better than row-col
% single programing model.  generacity.  communication optimality and strong scaling.

% Incompleteness -> sparse approximations dense problems, uses conventional sparse infrastructure, second order errors in matrix multiplication.  
% Often adhoc.

% Philosophy: Don't know which elements to drop, because only make sense in context.  It is very difficult to incorporate this context in a dropping 
% strategies. 

% Approximate Algebra, SpAMM Cauchy Schwarz occlusion, n-body approach to numerical linear algebra, first order errors in matrix multiplication.
% Based on Cauchy Schwarz inequality.

% We show that this first order error in algebraic context follows the functional analysis, and enables the consideration {\em and seperate treatment} of
% first order error flows in the analysis of complex maps. 



\begin{equation}
\mat{a} \ot \mat{b}=\mat{a}\cdot \mat{b} + \mat{\Delta}^{a\cdot b}_{\tau}
\end{equation}
where $\mat{\Delta}^{a \cdot b}_{\tau}$ is a deterministic (assymetric) first order variation cooresponding to the branch pattern set
by Cauchy-Schwarz occlusion, with length $\lVert \mat{\Delta}^{a \cdot b}_{\tau} \rVert \leq \tau \lVert \mat{a} \rVert  \lVert \mat{b} \rVert$.  
The opperator $\ot$ leads to a non-associative algebra with Lie bracket
\begin{equation}
\left[ \mat{a} , \mat{b} \right]_{\tau} = \mat{a} \ot \mat{b}-\mat{b} \ot \mat{a}  
=  \left[ \mat{a} , \mat{b} \right]
+ \mat{\Delta}^{a\cdot b}_{\tau} -\mat{\Delta}^{b\cdot a}_{\tau}.
\end{equation}

determined by the occlusion field.  Our challenge is to master the error flows of these occlusion fields under iteration, 
for ill-conditioned problems and with permisive values of $\tau$. 

\section{Newton Shulz Iteration}

\subsection{Idempotence }

\subsection{The Scaled Map}

\subsection{Alternative Formulations}
dual, stabilized and naive

\section{Occlusion Flows}

$\delta \mat{x}_k$ and $\delta \mat{z}_k$ arrize from itteration with $\ot$, and are deterministic 
flows away from the manifold of $\mat{s}$ determined by sensitivity of the NS iteration to these 
numerical insults. 


\begin{equation}
\delta \mat{x}^{\rm{naiv}}_k =   \delta  \widetilde{ \mat{z}}_{k} \cdot \mat{s} \cdot \widetilde{\mat{z}}_{k} 
                           +  \widetilde{\mat{z}}_{k} \cdot \mat{s} \cdot \! \delta \widetilde{\mat{z}}_{k} 
\end{equation}



\begin{equation}
\delta \mat{x}^{\rm{dual}}_k =   \delta  \widetilde{ \mat{y}}_{k} \cdot \widehat{\mat{z}}_{k} 
                           +  \widetilde{\mat{y}}_{k} \cdot \delta \widetilde{\mat{z}}_{k} 
\end{equation}


\begin{eqnarray}
\widetilde{\mat{x}}_k &=& f \left[\widetilde{\mat{z}}_{k-1} , \widetilde{\mat{x}}_{k-1} \right] \\ 
&=&
\tt{m} \left[ \widetilde{\mat{x}}_{k-1}\right] \cdot \widetilde{\mat{z}}^\dagger_{k-1}  
\cdot \mat{s} \cdot \widetilde{\mat{z}}_{k-1} \cdot \tt{m}\left[ \widetilde{\mat{x}}_{k-1} \right] 
\nonumber
\end{eqnarray}

\begin{equation}
\delta \mat{x}_k = {f}_{\delta \mat{z}_{k-1}}  \, \lVert \delta \mat{z}_{k-1} \rVert 
                              +  {f}_{\delta \mat{x}_{k-1}}   \, \lVert \delta \mat{x}_{k-1} \rVert 
                                                                                      + {\cal{O}} \left(  \tau^2 \right)
\end{equation}

generalized Gateaux differential

\begin{eqnarray}
f_{\delta \mat{z}_{k-1}} &=& \lim_{\tau \rightarrow 0} \frac{ f [ \mat{z}_{k-1} +\tau  \delta \widehat{\mat{z}}_{k-1}, \widetilde{\mat{x}}_{k-1} ]
-f [\, \mat{z}_{k-1}, \widetilde{\mat{x}}_{k-1} ]  }{\tau} \nonumber  \\[0.1cm] 
&=&{L}_{\widetilde{\mat{x}}_k}\left(\widetilde{\mat{z}}_{k} , \delta \widehat{\mat{z}}_{k-1} \right)  
\end{eqnarray}

\begin{eqnarray}
f_{\delta \mat{x}_{k-1}} &=& \lim_{\tau \rightarrow 0} \frac{ f [ \widetilde{\mat{z}}_{k-1}, \mat{x}_{k-1} + \tau \delta \widehat{\mat{x}}_{k-1} ]
-f [ \widetilde{\mat{z}}_{k-1}, \mat{x}_{k-1} ]  }{\tau} \nonumber  \\[0.1cm] 
&=&{L}_{\widetilde{\mat{x}}_k}\left(\widetilde{\mat{z}}_{k} , \delta \widehat{\mat{x}}_{k-1} \right)  
\end{eqnarray}

\begin{multline}
{L}_{\widetilde{\mat{x}}_k}\left(\widetilde{\mat{z}}_{k} , \delta \widehat{\mat{x}}_{k-1} \right) 
= \delta \widehat{\mat{x}}^\dagger_{k-1} \cdot   \tt{m}'\left[\mat{x}_{k-1} \right] \cdot 
\{\widetilde{\mat{z}}^\dagger_{k-1}  \cdot \mat{s} \cdot \widetilde{\mat{z}}_{k} \}  \\
+ \{ \widetilde{\mat{z}}^\dagger_{k} \cdot \mat{s} \cdot  \widetilde{\mat{z}}_{k-1} \} 
\cdot \tt{m}'\left[\mat{x}_{k-1} \right]  \cdot \delta \widehat{\mat{x}}_{k-1} 
\end{multline}

\begin{multline}
{L}_{\widetilde{\mat{x}}_k}\left(\widetilde{\mat{z}}_{k} , \delta \widehat{\mat{z}}_{k-1} \right) = 
\{ \tt{m}\left[\mat{x}_{k-1} \right]  \cdot \delta {\widehat{\mat{z}}}^\dagger_{k-1} 
 \cdot \mat{s} \} \cdot \widetilde{\mat{z}}_{k} \\
+\widetilde{\mat{z}}^\dagger_{k} \cdot \{ \mat{s} \cdot \delta {\widehat{\mat{z}}}_{k-1}
\cdot \tt{m}\left[\mat{x}_{k-1} \right]    \} 
\end{multline}

\begin{equation}
 \{ \widetilde{\mat{z}}^\dagger_{k} \cdot \mat{s} \cdot  \widetilde{\mat{z}}_{k-1} \} 
\rightarrow \mat{p}_+\left[\mat{s} \right]
\end{equation}

\begin{equation}
\{ \mat{s} \cdot \delta {\widehat{\mat{z}}}_{k-1}
\cdot \tt{m}\left[\mat{x}_{k-1} \right]    \} 
\rightarrow \mat{n}\left[\mat{s} \right]
\end{equation}

\section{Basis Set Ill-Conditioning in Electronic Structure}

\subsection{ 3,3 carbon nanotube with diffuse $sp$-function}
double exponential (Fig.)

\subsection{Water with triple zeta and double polarization}
Here's looking at you Jurg...

\section{Implementation}

\subsection{Methods}
FP, F08, OpenMP 4.0

\subsection{A Modified NS Map}

\subsection{$\delta \mat{x}_k$ and $\delta \mat{x}_k$ channels}
tau= Figure showing channels etc.  

\subsection{Convergence}
Map switching and etc based on TrX

\section{Experiments}

\subsection{Occlusion Flows}
\begin{figure}[h]
  \caption{equation...}
 \includegraphics[width=3.5in]{8x_33_nanotube_cond10_tau-5.eps}
 \includegraphics[width=3.5in]{8x_33_nanotube_cond10_tau-3.eps}
\end{figure}
\begin{figure}[h]
  \caption{equation...}
 \includegraphics[width=3.5in]{8x_33_nanotube_cond10_compare_errors.eps}
\end{figure}

\subsection{Comments}

\begin{multline}
 \delta {\mat{z}}_{k-1} \approx \Delta^{\widetilde{\mat{z}}_{k-2} \cdot \tt{m}\left[ \widetilde{\mat{x}}_{k-2}\right]}_\tau 
+ \mat{z}_{k-2} \cdot \tt{m}'\left[ \widetilde{\mat{x}}_{k-2}\right] \cdot \delta \mat{x}_{k-2} \\
+\delta \mat{z}_{k-2} \cdot \tt{m} \left[\widetilde{\mat{x}}_{k-2} \right] 
\end{multline}

\begin{multline}
\lVert \delta {\mat{z}}_{k-1} \rVert \lesssim
\lVert \mat{z}_{k-2} \rVert \left( \;  \tau \, \lVert \tt{m} \left[\widetilde{\mat{x}}_{k-2} \right]  \rVert \right.   \\ \left.
+ \; \lVert \delta {\mat{x}}_{k-2} \rVert   \lVert \tt{m}' \left[\widetilde{\mat{x}}_{k-2} \right] \rVert \; \right)
\end{multline}

\begin{equation}
\lVert \mat{z}_{k} \rVert  \rightarrow \sqrt{\kappa\left(\mat{s} \right)}
\end{equation}

\subsection{Scaling}

\subsection{Comments}
Pictures of the spamm structure

\section{Conclusion}

%%eg vs row-col picture.  Example of exact exchange w/DBSR 


\end{document}
