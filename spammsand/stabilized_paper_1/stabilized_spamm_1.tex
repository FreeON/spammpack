\documentclass[letterpaper,twocolumn,amsmath,amsfont,amssymb,english,aps,jcp,preprintnumbers,groupaddress,nofootinbib,tightenlines]{revtex4}

\usepackage{graphicx}
\usepackage{epstopdf}

%\documentclass[aps,prb,letterpaper,twocolumn,nofootinbib,showkeys]{revtex4-1}
%\documentclass[aps,amssymb,prl,letterpaper,twocolumn,nofootinbib,showkeys]{revtex4-1}

%\usepackage[backend=bibtex]{biblatex}

%    backend=biber,
%    style=authoryear,
%    natbib=true,
%    sortlocale=en_US,
%    url=false,
%    doi=true,f
%    eprint=false
%]{biblatex}
%\usepackage{hyperref}


\newcommand{\mat}[1]{\boldsymbol{#1}}
\newcommand{\mmat}[1]{\widetilde{\boldsymbol{#1}}}
\newcommand{\matT}[1]{\boldsymbol{#1}^\dagger}
\newcommand{\ot}{ {\scriptstyle \otimes}_{ \tau } }

%%\hypersetup{pdftitle={FreeON Project Report 1}}
%\hypersetup{pdfauthor={Matt Challacombe and Nicolas Bock}}
%\hypersetup{pdfsubject={A SpAMM Stabilized Newton Schulz Preconditioner: Fighting Error with Error}}

%\bibstyle{aipnum4-1}

\begin{document}

\title{On Stability of Newton Schulz Iterations in an Approximate Algebra}

\author{Matt Challacombe}
\email{matt.challacombe@freeon.org}
\homepage{http://www.freeon.org}
\affiliation{Theoretical Division, Los Alamos National Laboratory}

\author{Nicolas Bock}
\email{nicolasbock@freeon.org}
\homepage{http://www.freeon.org}
\affiliation{Theoretical Division, Los Alamos National Laboratory}

%\begin{abstract}
%Forward look
%\end{abstract}

\maketitle
\section{Introduction}

In many areas of application, finite correlations lead to matrices with decay properties.  By decay, we mean an approximate 
(perhaps bounded \cite{}) inverse relationship between matrix elements and an associated distance;  this may be a simple inverse 
exponential relationship between elements and the Cartesian distance between support functions, or it may 
involve a generalized distance, {\em e.g.}~ a statistical measure between strings.  
In electronic structure,  correlations manifest in decay properties of the gap shifted matrix 
sign function, as projector of the effective Hamiltonian (Fig.~\ref{figure1}).  
More broadly, matrix decay properties may coorespond to statistical matrices 
\cite{penrose1974,voit00,Anselin2003,Hardin2013,Krishtal2014}, including learned correlations in a 
generalized, non-orthogonal metric \cite{}. More broadly still, problems with local, non-orothogonal support 
are often solved with congruential transformations of the matrix inverse square root \cite{Lowdin56,naidu11} or a related factorization \cite{Krishtal2014};
these transformations correlate local support with a representation independent form, {\em eg.}~of the eigenproblem. 
Interestingly, the matrix sign function and the matrix inverse square root function are related by Higham's identity:
\begin{equation}
\rm{sign} \left( \begin{bmatrix} 0 & \mat{s}      \\ \mat{I}       & 0\end{bmatrix} \right)  =
                 \begin{bmatrix} 0 & \mat{s}^{1/2} \\ \mat{s}^{-1/2} & 0\end{bmatrix}  .
\end{equation}
A complete overivew of matrix function theory and computation is given in Higham's enjoyable reference \cite{Higham08}. 

A well conditioned matrix $\mat{s}$ may often correspond to matrix sign and inverse square root functions with rapid exponential decay, 
and be amenable to the sparse matrix approximation
$\bar{\mat{s}} = \mat{s}+ \mat{\epsilon}^{\mat{s}}_\tau$, where $\mat{\epsilon}^{\mat{s}}_\tau$ is the error introduced according to some  
criteria $\tau$.  Supporting this approximation are usefull bounds to matrix function elements \cite{Benzi99b, }.  
The criteria $\tau$ might be a drop-tolerence, 
$\epsilon^{\mat{s}}_{\tau} = \{-s_{ij}*\hat{\mat{e}}_i \, | \, |s_{ij}|<\tau \}$, a radial cutoff, 
$\epsilon^{\mat{s}}_{\tau} = \{-s_{ij}*\hat{\mat{e}}_i \, | \, \lVert \mat{r}_i - \mat{r}_j \rVert > \tau \}$, 
or some other approach to truncation, perhaps involving a sparsity pattern chosen {\em a priori}. 
Then, conventional computational kernels may be employed, such as the sparse general matrix-matrix multiply 
($\tt{SpGEMM}$) \cite{Gustavson78, Toledo97,challacombe00,bowler00}, yeiding fast solutions for multiplication rich iterations and a modulated fill in. 
These and related incomplete/inexact approaches to the computation of sparse approximate matrix functions often lead to ${\cal O}(n)$ 
algorithms, finding wide use in technologically important preconditioning schemes, the information sciences, electronic structure and many
other disciplines.  Comprehensive surveys of these methods in the numerical linear algebra are given by Benzi \cite{Benzi99,Benzi02}, and
by Bowler \cite{Bowler12} and Benzi \cite{Benzi13} for electronic structure.

\begin{figure}[t]\label{figure1}
 \includegraphics[width=3.5in]{decay_picture.png}
  \caption{Examples from electronic structure of decay for the spectral projector (gap shifted sign function) with respect to local (atomic) support.  
           Shown is decay for systems with correlations that are short (insulating water), medium (semi-conducting 4,3 nanotube), and 
           long (metalic 3,3 nanotube) ranged,  from exponential (insulating) to algebraic (metallic). }
\end{figure}

Because the truncated multiplication is controled only by absolute, addititve errors in the  product,   
\begin{equation} \label{sparseapprox}
\overline{ \mat{a} \cdot \mat{b} }\; = \; \mat{a}\cdot\mat{b} \; +\; \mat{\epsilon}^{\mat{a}}_\tau \cdot \mat{b} \;+\;
 \mat{a} \cdot \mat{\epsilon}^{\mat{b}}_\tau  \; + \;   {\mathcal O}(\tau^2)
\end{equation}
achieving sparse, stable and rapidly convergent iteration for ill-conditioned problems can be challenging \cite{}.  In cases of 
extreme degeneracy,  hierarchical semi-seperable (reduced rank) algorithms  can offer effective complexity reduction \cite{}.
However, many pratical cases are somewhere in-between sparse and meaningfully degenerate regimes; effectively dense but without
an exploitable reduction in rank.  This is the case in electronic structure for strong but non-metalic correlation, 
{\em e.g.}~towards the Mott transition \cite{}, and also in the case of local atomic support towards completeness \cite{Others, Hutter, Gigi}. 

\pagebreak
In this contribution, we consider an $N$-body approach to the approximation of matrix functions with decay, 
based on the quadtree data structure \cite{wise, samet} 
\begin{equation}
\mat{a}^i = \begin{bmatrix} \,  \mat{a}^{i+1}_{00} \, & \,  \mat{a}^{i+1}_{01} \,  \\[0.2cm]  \, \mat{a}^{i+1}_{10} \,  & \,\mat{a}^{i+1}_{11} \, \end{bmatrix} \, ,
\end{equation}
and orderings that are locality preserving \cite{}.  Orderings that preserve data locality are well developed in the
database theory \cite{}, providing fast spatial and metric querries.  
Locality enabled, fast data access is central to the $N$-Body approximation \cite{}, and an important problem 
for enterprise \cite{} and runtime systems \cite{}, with memory hierarchies becoming increasingly assynchronous and decentralized \cite{cache}.  
For matrices with decay, orderings that preserve locality lead to block-by-magnitude matrix structures with well 
segregated neighborhoods, inhabited by matrix elements of like size, and efficiently resolved by the quadtree data structure \cite{}.

With block-by-magnitude ordering of matrices $\mat{a}$ and $\mat{b}$, 
the Sparse Approximate Matrix Multiplication ($\tt SpAMM$) kernel,  $\ot$, carries out fast 
occlusion culling of insignifcant volumes in the product octree:
\begin{widetext}
\begin{equation}
\mat{a}^{i} \ot \mat{b}^{i} = 
\left\{
        \begin{array}{ll}
                 \emptyset \quad \tt{if}\quad \lVert \mat{a}^i \rVert \lVert \mat{b}^i \rVert < \tau \\[0.2cm]
                 \mat{a} ^i \cdot \mat{b}^i \quad  \tt{if}(i=\tt{leaf}) \\[0.2cm]
\begin{bmatrix} \mat{a}^{i+1}_{00} \ot \mat{b}^{i+1}_{00} +\mat{a}^{i+1}_{01} \ot \mat{b}^{i+1}_{10} \; , \; &
                \mat{a}^{i+1}_{00} \ot \mat{b}^{i+1}_{01} +\mat{a}^{i+1}_{01} \ot \mat{b}^{i+1}_{11}  \\[0.2cm] 
                \mat{a}^{i+1}_{00} \ot \mat{b}^{i+1}_{01} +\mat{a}^{i+1}_{01} \ot \mat{b}^{i+1}_{11} \; , \; & 
                \mat{a}^{i+1}_{00} \ot \mat{b}^{i+1}_{01} +\mat{a}^{i+1}_{01} \ot \mat{b}^{i+1}_{11}   
\end{bmatrix}  \quad \tt{else}
                \end{array}
              \right.  \, ,
\end{equation}
\end{widetext}
with errors linear in $\tau$ 
bounded by the sub-multiplicative norms $\lVert \cdot \rVert \equiv \lVert \cdot \rVert_F$ and the Cauchy-Schwarz inequality \cite{kahan,}.
In Ref.\cite{Challacombe2014}, we generalize this recursive task occlusion to the problem of Fock exchange.

The approximate $\tt SpAMM$ product is 
\begin{equation}
\widetilde{\mat{a}\cdot \mat{b}} \,  \equiv \, \mat{a} \ot \mat{b} \, 
  = \, \mat{a} \cdot \mat{b} + \mat{\Delta}^{a \cdot b}_{\tau} 
+ {\cal{O}} \left(  \tau^2 \right) \; ,
\end{equation}
with the culled contractions $\mat{\Delta}^{a \cdot b}_{\tau}$ obeying the $\tt SpAMM$ bound 
\begin{equation}\label{bound}
\lVert \mat{\Delta}^{a \cdot b}_{\tau} \rVert \, \leq \, \tau \, \lVert \mat{a} \rVert  \,  \lVert \mat{b} \rVert \, , 
\end{equation}
at each level of recursion.  This makes $\ot$ {\em stable}, as defined by Demmel, Dumitriu and Holz (see Eq.(1), Ref.~\cite{Demmel07}). 
However, instead of the roundoff error, we are concerned with the deterministic $\tt SpAMM$ error,  which 
leads to a non-associative algebra and error flows with properties of the Lie bracket
\begin{equation}
\widetilde{\left[ \mat{a} , \mat{b} \right]} \equiv \mat{a} \ot \mat{b}-\mat{b} \ot \mat{a}  
=  \left[ \mat{a} , \mat{b} \right]
+ \mat{\Delta}^{a\cdot b}_{\tau} -\mat{\Delta}^{b\cdot a}_{\tau} \,.
\end{equation}
The interesting group theory associated with the construction of matrix functions is developed by Higham, Mackey, Mackey and T 
in Ref.\cite{}.  
%In this contibution, we consider only the magnitude of these error flows.     

$\tt SpAMM$ is similar to compressed kernels for sketching the matrix product \cite{Kutzkov2012, Pagh2013}. 
spatial join! Paugh.  MAD.  etc
 Instead of
the FFT however, compression is achieved through a recursive two-sided metric querry on the $\tt SpAMM$ bound, Eq.~(\ref{bound}),
which may exprerience acceleration through localiztion effects (lensing) in the $ijk$ (octree) space, as demonstrated shortly. 
These localization techniques are, in most cases, complimentary with other compresive technologies, including methods based on 
hierarchical semi-seperability and also methods for fast kernel summation \cite{}. 
In addition to compression by occlusion, octree locality is important for the communication optimality of 
$n$-body methods \cite{Warren Salmon, Yellik} that may be achieved by minimal locally essential trees \cite{}.    
Finally, its interesting to note that this data-centric, {\em locality as compression} approach is incompatible with 
randomization methods that employ homogenization to well condition matrices 
\cite{pan, DiahLi and Parket Scott},  and also to achieve domain decomposition of the $\tt SpGEMM$, {\em e.g.} 
using the conventional SUMMA \cite{}.

\section{First Order Newton-Shulz Iteration}

There are two common, first order NS iterations; the sign iteration and the square root iteration, related by the square, $\mat{I}\left( \cdot \right)= {\rm sign}^2\left( \cdot \right) $.  These equivalent iterations converge linearly at first, then enters a basin of stability marked by super-linear convergence. 
Our interest is to access this basin with the most permisive $\tau$ possible, building a foundation for future refinement
%, {\em e.g.} 
at a reduced cost and with a higher precision ($\tau \rightarrow 0$) \cite{MChallacombe16}.

\subsection{Sign iteration}
For the NS sign iteration, this basin is marked by a behavioral change in the
difference $\delta \mat{X}_k = \widetilde{\mat{X}}_k -\mat{X}_k = {\rm sign} \left(\mat{X}_{k-1}+\delta \mat{X}_{k-1} \right)
-{\rm sign} \left(\mat{X}_{k-1} \right)$, where $\delta \mat{X}_{k-1}$ is some previous error.
The change in behavior is associated with the onset of idempotence and the bounded eigenvalues of ${\rm sign}'\left( \cdot \right)$, leading to stable 
iteration when ${\rm sign}'\left( \mat{X}_{k-1} \right) \delta \mat{X}_{k-1} < 1 $.  
Global perturbative bounds on this iteration have been derived by Bai and Demmel \cite{Bai98usingthe}, while
Byers, He and Mehrmann \cite{} developed assymptotic bounds.  The automatic stability of sign iteration is a well developed theme in Ref.\cite{Higham08}.

\subsection{Square root iteration}
In this work, we are concerned with  resolution of the identity \cite{}
\begin{equation}
\mat{I} \left( \mat{s} \right) =\mat{s}^{1/2} \cdot \mat{s}^{-1/2} \, ,
\end{equation}
and the cooresponding cannonical (dual) square root iteration \cite{};
\begin{eqnarray}\label{cannonical}
\mat{y}_k &\leftarrow& h_\alpha \left[ \mat{y}_{k-1} \cdot \mat{z}_{k-1} \right] \cdot \mat{y}_{k-1}  \nonumber \\ 
\mat{z}_k &\leftarrow& \mat{z}_{k-1} \cdot h_\alpha \left[ \mat{y}_{k-1} \cdot \mat{z}_{k-1} \right] \; ,
\end{eqnarray}
with eigenvalues in the proper domain aggregated towards 0 or 1 by the 
NS map $h_\alpha[\mat{x}]=\frac{\sqrt{\alpha}}{2} \left(3-\alpha \mat{x} \right)$  \cite{}. Then,
starting with $\mat{z}_0=\mat{I}$ and $\mat{x}_0=\mat{y}_0=\mat{s}$,
${\mat{y}}_k \rightarrow \mat{s}^{1/2}$, ${\mat{z}}_k \rightarrow \mat{s}^{-1/2}$ and ${\mat{x}}_k \rightarrow {\mat{I}}$.
As in the case of sign iteration, this dual iteration was shown by Higham, Mackey,  Mackey and Tisseur \cite{Higham2005} 
to remain bounded in the superlinear regime, by idempotent Frechet derivatives about the fixed point $\left(\mat{s}^{1/2},\mat{s}^{-1/2}\right)$,
in the direction $\left( \delta \mat{y}_{k-1} , \delta \mat{z}_{k-1} \right)$:
\begin{eqnarray}
\delta \mat{y}_k &=& \frac{1}{2} \delta \mat{y}_{k-1} - \frac{1}{2} \mat{s}^{1/2} \cdot \delta \mat{z}_{k-1} \cdot \mat{s}^{1/2} \\
\delta \mat{z}_k &=& \frac{1}{2} \delta \mat{z}_{k-1} - \frac{1}{2} \mat{s}^{-1/2} \cdot \delta \mat{y}_{k-1} \cdot \mat{s}^{-1/2} \;.
\end{eqnarray}
In this contribution, we consider another aspect of convergence, namely the (hopefully) linear approach towards stability of the iteration 
\begin{equation}
\widetilde{\mat{x}}_k \leftarrow 
 \widetilde{\mat{y}}_k \left( \widetilde{\mat{x}}_{k-1} \right)
\, \ot \, \widetilde{\mat{z}}_k \left( \widetilde{\mat{x}}_{k-1} \right) \, ,
\end{equation}
made difficult by ill-conditioning and a sketchy $\ot$.

\subsubsection{the NS map}
Initially, $h'_\alpha$ at the smallest eigenvalue $x_0$ controls the rate of progress towards idempotence.  
As recently shown by Jie and Chen \cite{Chen2014}, for very ill-conditioned problems, a factor of two reduction in the number of NS steps can be achieved 
by chosing $\alpha \sim 2.85$, which is at the edge of stability.   As argued by Pan and Schreiber \cite{Pan1991},
Jie and Chen \cite{Chen2014}, switching or damping the scaling factor towards $\alpha=1$ at convergence is important, shifting emphasis away from the 
behavior of $x_0$ towards {\em e.g.} $x_i \in [0.01,1]$, emphasizing overall convergence of the broad distribution \cite{Pan and Scriber}. 
In an approximate algebra like $\tt SpAMM$, the potential for eigenvalues to fluctuate out of the domain of convergence must be considered.
This is addressed in Section.~\ref{}. 

\subsubsection{stability and ill-conditioning}

Agressive scaling can reduce stability of the iteration due to the larger derivative $h'_\alpha$.  Also, the previous error, 
$\delta \mat{x}_{k-1}$, may be to large, {\em e.g.} due to a too large value of $\tau$,  leading to 
the unbounded (exponential) accumulation of errors $\delta \mat{x}_k > 1 $.  To first order, stability of the 
iteration is controled by the Frechet derivatives contributing to $\delta \mat{x}_k$.    For ill-conditioned problems
and the $\ot$ kernel, these derivatives can behave differently amongst nominally equivalent implementations, for example 
formulations based on the assumption of a commuting algebra.  In this contribution, we are interested in the
cannonical (dual) square root iteration, Eq.~\ref{cannonical}, as well as the ``stabilized'' version 
with $\mat{y}^{\rm stab}_k = \mat{z}^\dagger_k \cdot \mat{s}$.  

%For full precision, $\mat{y}^{\rm stab}_k$ is equivalent to 
%$\mat{y}^{\rm dual}_k$ \cite{higham2006}.  However, in 

\subsubsection{lensing}

A feature of square root iteration with the $\ot$ kernel is localization of the culled octree 
with convergence towards identity, $\widetilde{\mat{x}}_k \leftarrow \mat{I}\left( \widetilde{\mat{x}}_{k-1} \right)$.  In the product 
$\widetilde{\mat{y}}_k \ot \widetilde{\mat{z}}_k$, large and small eigenvalues, as well as norms, 
are recursively brought to unity along the diagonal $i=k$, and within the cube $ijk$.
Because the $\tt SpAMM$ error obeys the multiplicative Cauchy-Schwarz bound, Eq.~(),  the 
cooresponding culled-octree follows the $i=k$ plane\footnote{as well as the $i=j$ plane.}, resolving 
the {\em relative} error in identity to within $\tau$.   
This effect is shown in Figure \ref{x_lensing}.   We call this identity related,  plane-wise concentration of the culled octree 
{\em lensing}.  Ideally, lensing cooresponds to the formation of two-dimensional surfaces, crossing along the cube diagonal, with attenuation 
off the diagonal in the case of decay.  This is a complexity reduction relative to the naive (full) volume of the cube,
and also relative to sparsification strategies that preserve only absolute errors, as in Eq.~\ref{}.  

\section{Modeling the ${\tt SpAMM}$ NS Iteration}

There are multiple effects we wish to model in the $\ot$-approximate square root iteration.  Primarily, 
classical first order errors  in $\delta \mat{x}_k$ are controled by the 
Fr\'{e}chet derivatives \cite{} ${\mat{x}}_{\delta \mat{y}_{k-1}}$ and ${\mat{x}}_{\delta \mat{z}_{k-1}}$ 
at $\mat{x}_{k-1}$ and along the unit directions  $\delta \widehat{\mat{y}}_{k-1}$ and $\delta \widehat{\mat{z}}_{k-1}$;
\begin{equation}
\delta \mat{x}_k = \,  { \mat{x}}_{\delta \widehat{\mat{y}}_{k-1}}  \, {\scriptstyle \times} \, \delta y_{k-1} 
                 \, + \,  { \mat{x}}_{\delta \widehat{\mat{z}}_{k-1}}  \, {\scriptstyle \times} \, \delta z_{k-1}  \, ,
\end{equation}
with previous displacements $\delta y_{k-1} = \lVert \delta \mat{y}_{k-1} \rVert$  and  $\delta z_{k-1}=\lVert \delta \mat{z}_{k-1} \rVert$.
However, these displacements are augmented by the $\tt SpAMM$ error at each step.   
%In the case of the stabilized iteration, we have for example $\delta \mat{y}^{\rm stab}_{k-1}= \mat{\Delta}^{\mat{z}^\dagger \cdot \mat{s}}_\tau$.
Also, second order effects can develop (due to ill-conditioning), which invalidate $\mat{x}_{k-1}$ as reference for 
first order expansion.   
 
\subsection{Fr\'{e}chet Analyses}

Our model is based on classical Fr\'{e}chet analyses \cite{} of the matrix function 
\begin{multline}
 \mat{x} \left( \mat{y}_{k-1}, \mat{z}_{k-1}  \right)= h_\alpha \left[ \mat{y}_{k-1} \cdot \mat{z}_{k-1} \right] \cdot  \mat{y}_{k-1}  \\
 \cdot  \mat{z}_{k-1} \cdot h_\alpha \left[ \mat{y}_{k-1} \cdot \mat{z}_{k-1} \right] \, ,
\end{multline}
with all previous errors mapped to $\delta \mat{y}_{k-1}$ and $\delta \mat{z}_{k-1}$ by the complicated logistics of $h_\alpha$ and $\ot$. 
Later, we will consider the also the  first $\tt SpAMM$ error at step $k$.  At present however, we turn to the Fr\'{e}chet derivatives \cite{}:
 \begin{eqnarray}
  \mat{x}_{\delta \widehat{ \mat{y}}_{k-1}} &=& \lim_{\tau \rightarrow 0} \frac{ \mat{x} \left( \mat{y}_{k-1} + \tau \delta \widehat{\mat{y}}_{k-1} ,  {\mat{z}}_{k-1}  \right)
 -\mat{x} \left( \mat{y}_{k-1}  ,  {\mat{z}}_{k-1} \right)  }{\tau}  \nonumber  \\[0.1cm] 
&=&   \mat{y}_{\delta \widehat{ \mat{y}}_{k-1}} \cdot \mat{z}_{k}  + \mat{y}_{k}  \cdot \mat{z}_{\delta \widehat{ \mat{y}}_{k-1}} 
 \end{eqnarray}
and 
 \begin{eqnarray}
 \mat{x}_{\delta \widehat{ \mat{z}}_{k-1}} &=& \lim_{\tau \rightarrow 0} 
\frac{ \mat{x} \left( {\mat{y}}_{k-1} , \mat{z}_{k-1} +\tau  \delta \widehat{\mat{z}}_{k-1} \right)
     - \mat{x} \left(\, {\mat{y}}_{k-1} , \mat{z}_{k-1} \right)  }{\tau} \nonumber  \\[0.1cm] 
&=& \mat{y}_{\delta \widehat{ \mat{z}}_{k-1}} \cdot \mat{z}_{k}  + \mat{y}_{k}  \cdot \mat{z}_{\delta \widehat{ \mat{z}}_{k-1}} \, . 
 \end{eqnarray}

%\pagebreak

\subsubsection{$\mat{x}_{\delta \widehat{ \mat{y}}_{k-1}}$}

For $\mat{x}_{\delta \widehat{ \mat{y}}_{k-1}}$ we have
\begin{multline}
 \mat{y}_{\delta \widehat{ \mat{y}}_{k-1}} = h_\alpha \left[ \mat{x}_{k-1} \right] \cdot \delta \widehat{\mat{y}}_{k-1} \\ 
                                     + h'_\alpha \cdot \delta \widehat{\mat{y}}_{k-1} \cdot \mat{z}_{k-1} \cdot \mat{y}_{k-1}
\end{multiline}
and
\begin{equation}
 \mat{z}_{\delta \widehat{ \mat{y}}_{k-1}} =  \mat{z}_{k-1} \cdot h'_\alpha \delta \widehat{\mat{y}}_{k-1} \cdot \mat{z}_{k-1}
\end{equation}
yeilding 
\begin{multline} 
 \mat{x}_{\delta \widehat{ \mat{y}}_{k-1}} = h_\alpha \left[ \mat{x}_{k-1} \right]  \cdot \delta \widehat{\mat{y}}_{k-1} \cdot \mat{z}_{k} \\  
+  h'_\alpha  \delta \widehat{\mat{y}}_{k-1} \cdot \mat{z}_{k-1} \cdot  \mat{y}_{k-1} \cdot  \mat{z}_{k} \\ 
+ \mat{y}_{k} \cdot \mat{z}_{k-1} \cdot h'_\alpha \delta \widehat{\mat{y}}_{k-1} \cdot \mat{z}_{k-1} 
\end{multiline}
Closer to convergence, 
\begin{eqnarray} 
 \mat{x}_{\delta \widehat{ \mat{y}}_{k-1}} & \rightarrow& \delta \widehat{\mat{y}}_{k-1} \cdot \mat{z}_{k} + 2  \delta \widehat{\mat{y}}_{k-1} \cdot \mat{z}_{k-1} \\
$\rightarrow$ \delta \mat{y}_{k-1} \cdot \left( \mat{z}_k-\mat{z}_{k-1} \right)
\end{eqnarray}

%\pagebreak

\subsubsection{$\mat{x}_{\delta \widehat{\mat{z}}_{k-1}}$}

For $\mat{x}_{\delta \widehat{ \mat{z}}_{k-1}}$ we have
\begin{equation}
 \mat{y}_{\delta \widehat{ \mat{z}}_{k-1}} =  {\mat{y}}_{k-1} \cdot  h'_\alpha \delta \widehat{ \mat{z}}_{k-1} \cdot  \mat{y}_{k-1} 
\end{equation}
and
\begin{multline}
 \mat{z}_{\delta \widehat{ \mat{z}}_{k-1}} = \delta \widehat{\mat{z}}_{k-1} \cdot   h_\alpha \left[ \mat{x}_{k-1} \right] \\ 
                                     + \mat{z}_{k-1} \cdot {\mat{y}}_{k-1} \cdot h'_\alpha \delta \widehat{\mat{z}}_{k-1}  \; ,
\end{multiline}
yeilding 
\begin{multline} 
 \mat{x}_{\delta \widehat{ \mat{z}}_{k-1}} =  {\mat{y}}_{k-1} \cdot  h'_\alpha \delta \widehat{ \mat{z}}_{k-1} \cdot  \mat{y}_{k-1}  \cdot \mat{z}_{k} \\
\qquad + \mat{y}_k \cdot  \delta \widehat{\mat{z}}_{k-1} \cdot   h_\alpha \left[ \mat{x}_{k-1} \right] \\ 
\qquad \qquad +  \mat{y}_{k} \cdot  \mat{z}_{k-1} \cdot {\mat{y}}_{k-1} \cdot h'_\alpha \delta \widehat{\mat{z}}_{k-1}
\end{multiline}


Closer to convergence, 
\begin{equation} 
 \mat{x}_{\delta \widehat{ \mat{y}}_{k-1}} \rightarrow \delta \widehat{\mat{y}}_{k-1} \cdot \mat{z}_{k} + 2  \delta \widehat{\mat{y}}_{k-1} \cdot \mat{z}_{k-1} \,.
\end{equation}

%\pagebreak 

\subsubsection{$\mat{x}_{\delta \widehat{\mat{x}}_{k-1}}$}
\begin{multline} \label{fyzdx}
\mat{x}_{\delta \widehat{\mat{x}}_{k-1}} 
= h'_\alpha  \, \delta \widehat{\mat{x}}_{k-1} \cdot    
{\mat{y}}_{k-1}  \cdot \widetilde{\mat{z}}_{k}   \\
+ {\mat{y}}_{k} \cdot \widetilde{\mat{z}}_{k-1}  
\cdot  \delta \widehat{\mat{x}}_{k-1} \,  h'_\alpha  \;.
\end{multline}


\begin{equation}
\{ \mat{s} \cdot \delta {\widehat{\mat{z}}}_{k-1}
\cdot \tt{m}\left[\mat{x}_{k-1} \right]    \} 
\rightarrow \mat{n}\left[\mat{s} \right]
\end{equation}

Ideally, a $\tau$ can be found yeilding fast computation with precision sufficient to gain the basin of stability.  From the
preconditioned state then, additional corrections can be made to the residual at little additional 
cost. \footnote{The ability to correct back to the argumental basis is sometimes refered to as 
the ``variational'' property of early ``spectral projection as optimization'' techniques, with 
gradients retaining proximity to the basis \cite{}.}






\section{Implementation}

\subsection{Methods}
FP, F08, OpenMP 4.0

\subsection{A Modified NS Map}

\subsection{$\delta \mat{x}_k$ and $\delta \mat{x}_k$ channels}
tau= Figure showing channels etc.  

\subsection{Convergence}
Map switching and etc based on TrX


\section{Ill-Conditioned Support}

\subsection{ 3,3 carbon nanotube with diffuse $sp$-function}
double exponential (Fig.)

\subsection{Water with triple zeta and double polarization}
Here's looking at you Jurg...

\section{Experiments}

 \begin{figure}[h] \label{markofzorro}
 \fbox{ \includegraphics[width=1.in,trim={11.3cm 0cm 11.3cm 0cm},clip]{tube_dual_c6_x128_b64/x_19_scene1.png}} 
 \fbox{ \includegraphics[width=2.in,trim={2cm 0cm 10cm 0cm},clip]{tube_dual_c6_x128_b64/x_19_scene2.png}} 
 \caption{Views of the $\tau =0.03$ sign occlusion surface, for the 
 128x~u.c.~nanotube, at $\sim {14k \times 14k}$ and $\kappa(\mat{s})=10^6$. 
 This surface envelopes the $ijk$ volume of the $\ot$ kernel,  
 cooresponding to the unscaled dual iteration step $\mmat{x}_{19} \leftarrow \mmat{y}_{19} \ot \mmat{z}_{19} $ at $b=64$, $\tau=0.03$ and
 $\tau_y=10^{-3} \, \tau $.  The first pannel looks straight down the cube-diagonal $i=j=k$, from the upper bound towards (1,1,1).
 Remarkably, this surface forms an elongated $\times$, closely following intersection of the $i=j$  and $i=k$ planes 
 along the cube-diagonal. The second pannel looks along the cube-diagonal, with the upper bound at upper left, and (1,1,1) at lower right.}
 \end{figure}

In this section, we present  numerical experiments that highlight the effects of 
ill-conditioning, dimensionality, and the stability of different first order NS approaches to iteration with $\tt SpAMM$. 
We turn first to complexity reduction for $\ot$ in the basin of stability,  where we find a novel, compressive 
effect in the product octree.  This effect is shown in Fig.~\ref{markofzorro},  
for unscaled, inverse square root duals iteration, Eqs.~(\ref{dualsiteration}), on the 3,3 carbon 
nanotube metric at $\kappa=10^6$.  

In this example, the $\tt SpAMM$ octree culled from the $ijk$-cube is outlined by its occlusion surface, enclosing 
a volume that closely follows the $i=j$ and $i=k$ planes, forming an $\times$.  The banded distribution
of large norms along  matrix diagonals leads to cube-diagonal dominance, with plane-following 
a consequence of moderate ill-conditioning,  large norms along the plane-diagonals and their overlap in $ijk$
via the multiplicative bound, Eq.~(\ref{bound}). The tightness of this bound, and the compression gained relative
to  methods that control only the absolute error, {\em e.g.} as given by Eq.~(\ref{sparseapprox}), will hopefully
be quantified in future work. 





\begin{figure}[h]
  \caption{equation...}
\fbox{ \includegraphics[width=1.5in,trim={6cm 6cm 6cm 6cm},clip]{y_15_water.png}} 
\fbox{ \includegraphics[width=1.5in,trim={6cm 6cm 6cm 6cm},clip]{z_15_water.png}}
\end{figure}



\begin{figure}[h]
  \caption{equation...}
%%\fbox{ \includegraphics[width=1.5in,trim={6cm 6cm 6cm 6cm},clip]{y_water_to_duals.png}} 
%%\fbox{ \includegraphics[width=1.5in,trim={6cm 6cm 6cm 6cm},clip]{z_water_to_duals.png}}
%\fbox{ \includegraphics[width=1.5in,trim={6cm 6cm 6cm 6cm},clip]{y_water_to_duals.png}} 
\fbox{ \includegraphics[width=3in,trim={8cm 0.1cm 8cm 0.1cm},clip]{z_water_to_duals_scn1.png}}
\end{figure}




\subsection{Occlusion Error Flows}
\begin{figure}[h]
  \caption{equation...}
 \includegraphics[width=3.5in]{8x_33_nanotube_cond10_tau-5.eps}
 \includegraphics[width=3.5in]{8x_33_nanotube_cond10_tau-3.eps}
\end{figure}
\begin{figure}[h]
  \caption{equation...}
 \includegraphics[width=3.5in]{8x_33_nanotube_cond10_compare_errors.eps}
\end{figure}

\subsection{Comments}

\begin{multline}
 \delta {\mat{z}}_{k-1} \approx \Delta^{\widetilde{\mat{z}}_{k-2} \cdot \tt{m}\left[ \widetilde{\mat{x}}_{k-2}\right]}_\tau 
+ \mat{z}_{k-2} \cdot \tt{m}'\left[ \widetilde{\mat{x}}_{k-2}\right] \cdot \delta \mat{x}_{k-2} \\
+\delta \mat{z}_{k-2} \cdot \tt{m} \left[\widetilde{\mat{x}}_{k-2} \right] 
\end{multline}

\begin{multline}
\lVert \delta {\mat{z}}_{k-1} \rVert \lesssim
\lVert \mat{z}_{k-2} \rVert \left( \;  \tau \, \lVert \tt{m} \left[\widetilde{\mat{x}}_{k-2} \right]  \rVert \right.   \\ \left.
+ \; \lVert \delta {\mat{x}}_{k-2} \rVert   \lVert \tt{m}' \left[\widetilde{\mat{x}}_{k-2} \right] \rVert \; \right)
\end{multline}

\begin{equation}
\lVert \mat{z}_{k} \rVert  \rightarrow \sqrt{\kappa\left(\mat{s} \right)}
\end{equation}






\subsection{Comments}

\subsection{Found Contraction}

\subsection{Comments}
Pictures of the spamm structure

\section{Conclusion}

%%eg vs row-col picture.  Example of exact exchange w/DBSR 

\bibliography{MatrixFunctions}

\end{document}
