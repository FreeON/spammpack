\documentclass[letterpaper,twocolumn,amsmath,amsfont,amssymb,english,aps,jcp,preprintnumbers,groupaddress,nofootinbib,tightenlines]{revtex4}

\usepackage{graphicx}


%\documentclass[aps,prb,letterpaper,twocolumn,nofootinbib,showkeys]{revtex4-1}
%\documentclass[aps,amssymb,prl,letterpaper,twocolumn,nofootinbib,showkeys]{revtex4-1}

%\usepackage[backend=bibtex]{biblatex}

%    backend=biber,
%    style=authoryear,
%    natbib=true,
%    sortlocale=en_US,
%    url=false,
%    doi=true,f
%    eprint=false
%]{biblatex}
%\usepackage{hyperref}


\newcommand{\mat}[1]{\boldsymbol{#1}}
\newcommand{\matT}[1]{\boldsymbol{#1}^\dagger}
\newcommand{\ot}{ {\scriptstyle \otimes}_{ \tau } }

%%\hypersetup{pdftitle={FreeON Project Report 1}}
%\hypersetup{pdfauthor={Matt Challacombe and Nicolas Bock}}
%\hypersetup{pdfsubject={A SpAMM Stabilized Newton Schulz Preconditioner: Fighting Error with Error}}

%\bibstyle{aipnum4-1}

\begin{document}

\title{On Stability of Newton Schulz Iterations in an Approximate Algebra}

\author{Matt Challacombe}
\email{matt.challacombe@freeon.org}
\homepage{http://www.freeon.org}
\affiliation{Theoretical Division, Los Alamos National Laboratory}

\author{Nicolas Bock}
\email{nicolasbock@freeon.org}
\homepage{http://www.freeon.org}
\affiliation{Theoretical Division, Los Alamos National Laboratory}

%\begin{abstract}
%Forward look
%\end{abstract}

\maketitle
\section{Introduction}

In many areas of application, finite correlations lead to matrices with decay properties.  By decay, we mean an approximate 
(perhaps bounded \cite{}) inverse relationship between matrix elements and an associated distance;  this may be a simple inverse 
exponential relationship between elements and the Cartesian distance between support functions, or it may 
involve a generalized distance, {\em e.g.}~ a statistical measure between strings.  
In electronic structure,  correlations manifest in decay properties of the gap shifted matrix 
sign function, as projector of the effective Hamiltonian (Fig.~\ref{figure1}).  
More broadly, matrix decay properties may coorespond to statistical matrices 
\cite{penrose1974,voit00,Anselin2003,Hardin2013,Krishtal2014}, including learned correlations in a 
generalized, non-orthogonal metric \cite{}. More broadly still, problems with local, non-orothogonal support 
are often solved with congruential transformations of the matrix inverse square root \cite{Lowdin56,naidu11} or a related factorization \cite{Krishtal2014};
these transformations correlate local support with a representation independent form, {\em eg.}~of the eigenproblem. 
Interestingly, the matrix sign function and the matrix inverse square root function are related by Higham's identity:
\begin{equation}
\rm{sign} \left( \begin{bmatrix} 0 & \mat{s}      \\ \mat{I}       & 0\end{bmatrix} \right)  =
                 \begin{bmatrix} 0 & \mat{s}^{1/2} \\ \mat{s}^{-1/2} & 0\end{bmatrix}  .
\end{equation}
A complete overivew of matrix function theory and computation is given in Higham's enjoyable reference \cite{Higham08}. 

A well conditioned matrix $\mat{s}$ may often correspond to matrix sign and inverse square root functions with rapid exponential decay, 
and be amenable to the sparse matrix approximation
$\bar{\mat{s}} = \mat{s}+ \mat{\epsilon}^{\mat{s}}_\tau$, where $\mat{\epsilon}^{\mat{s}}_\tau$ is the error introduced according to some  
criteria $\tau$.  Supporting this approximation are usefull bounds to matrix function elements \cite{Benzi99b, }.  
The criteria $\tau$ might be a drop-tolerence, 
$\epsilon^{\mat{s}}_{\tau} = \{-s_{ij}*\hat{\mat{e}}_i \, | \, |s_{ij}|<\tau \}$, a radial cutoff, 
$\epsilon^{\mat{s}}_{\tau} = \{-s_{ij}*\hat{\mat{e}}_i \, | \, \lVert \mat{r}_i - \mat{r}_j \rVert > \tau \}$, 
or some other approach to truncation, perhaps involving a sparsity pattern chosen {\em a priori}. 
Then, conventional computational kernels may be employed, such as the sparse general matrix-matrix multiply 
($\tt{SpGEMM}$) \cite{Gustavson78, Toledo97,challacombe00,bowler00}, yeiding fast solutions for multiplication rich iterations and a modulated fill in. 
These and related incomplete/inexact approaches to the computation of sparse approximate matrix functions often lead to ${\cal O}(n)$ 
algorithms, finding wide use in technologically important preconditioning schemes, the information sciences, electronic structure and many
other disciplines.  Comprehensive surveys of these methods in the numerical linear algebra are given by Benzi \cite{Benzi99,Benzi02}, and
by Bowler \cite{Bowler12} and Benzi \cite{Benzi13} for electronic structure.

\begin{figure}[t]\label{figure1}
 \includegraphics[width=3.5in]{decay_picture.png}
  \caption{Examples from electronic structure of decay for the spectral projector (gap shifted sign function) with respect to local (atomic) support.  
           Shown is decay for systems with correlations that are short (insulating water), medium (semi-conducting 4,3 nanotube), and 
           long (metalic 3,3 nanotube) ranged,  from exponential (insulating) to algebraic (metallic). }
\end{figure}

Because the truncated multiplication is controled only by absolute, addititve errors in the  product,   
\begin{equation}
\overline{ \mat{a} \cdot \mat{b} }\; = \; \mat{a}\cdot\mat{b} \; +\; \mat{\epsilon}^{\mat{a}}_\tau \cdot \mat{b} \;+\;
 \mat{a} \cdot \mat{\epsilon}^{\mat{b}}_\tau  \; + \;   {\mathcal O}(\tau^2)
\end{equation}
achieving sparse, stable and rapidly convergent iteration for ill-conditioned problems can be challenging \cite{}.  In cases of 
extreme degeneracy,  hierarchical semi-seperable (reduced rank) algorithms  can offer effective complexity reduction \cite{}.
However, many pratical cases are somewhere in-between sparse and meaningfully degenerate regimes; effectively dense but without
an exploitable reduction in rank.  This is the case in electronic structure for strong but non-metalic correlation, 
{\em e.g.}~towards the Mott transition \cite{}, and also in the case of local atomic support towards completeness \cite{Others, Hutter, Gigi}. 

\pagebreak
In this contribution, we consider an $N$-body approach to the approximation of matrix functions with decay, 
based on the quadtree data structure \cite{wise, samet} 
\begin{equation}
\mat{a}^i = \begin{bmatrix} \,  \mat{a}^{i+1}_{00} \, & \,  \mat{a}^{i+1}_{01} \,  \\[0.2cm]  \, \mat{a}^{i+1}_{10} \,  & \,\mat{a}^{i+1}_{11} \, \end{bmatrix} \, ,
\end{equation}
and orderings that are locality preserving \cite{}.  Orderings that preserve data locality are well developed in the
database theory \cite{}, providing fast spatial and metric querries.  
Locality enabled, fast data access is central to the $N$-Body approximation \cite{}, and an important problem 
for enterprise \cite{} and runtime systems \cite{}, with memory hierarchies becoming increasingly assynchronous and decentralized \cite{cache}.  
For matrices with decay, orderings that preserve locality lead to block-by-magnitude matrix structures with well 
segregated neighborhoods, inhabited by matrix elements of like size, and efficiently resolved by the quadtree data structure \cite{}.
Then, the Sparse Approximate Matrix Multiplication ($\tt SpAMM$) kernel, $\ot$, carries out fast, recursive occlusion culling of 
insignifcant contractions, with errors bounded by sub-multiplicative norms and the Cauchy-Schwarz inequality \cite{}:
\begin{widetext}
\begin{equation}
\mat{a}^{i} \ot \mat{b}^{i} = 
\left\{
        \begin{array}{ll}
                 \emptyset \quad \tt{if}\quad \lVert \mat{a}^i \rVert \lVert \mat{b}^i \rVert < \tau \\[0.2cm]
                 \mat{a} ^i \cdot \mat{b}^i \quad  \tt{if}(i=\tt{leaf}) \\[0.2cm]
\begin{bmatrix} \mat{a}^{i+1}_{00} \ot \mat{b}^{i+1}_{00} +\mat{a}^{i+1}_{01} \ot \mat{b}^{i+1}_{10} \; , \; &
                \mat{a}^{i+1}_{00} \ot \mat{b}^{i+1}_{01} +\mat{a}^{i+1}_{01} \ot \mat{b}^{i+1}_{11}  \\[0.2cm] 
                \mat{a}^{i+1}_{00} \ot \mat{b}^{i+1}_{01} +\mat{a}^{i+1}_{01} \ot \mat{b}^{i+1}_{11} \; , \; & 
                \mat{a}^{i+1}_{00} \ot \mat{b}^{i+1}_{01} +\mat{a}^{i+1}_{01} \ot \mat{b}^{i+1}_{11}   
\end{bmatrix}  \quad \tt{else}
                \end{array}
              \right.  \, ,
\end{equation}
\end{widetext}
with $\lVert \cdot \rVert \equiv \lVert \cdot \rVert_F$, because computability \cite{kahan}. 

The total error associated with $\ot$ is 
\begin{equation}
\mat{\Delta}^{a\cdot b}_{\tau}=\mat{a} \ot \mat{b}-\mat{a}\cdot \mat{b} \, ,
\end{equation}
corresponding to a {\em volume} of occluded contractions, and obeying the multiplicative bound 
\begin{equation}
\lVert \mat{\Delta}^{a \cdot b}_{\tau} \rVert \, \leq \, \tau \, \lVert \mat{a} \rVert  \,  \lVert \mat{b} \rVert \, ,
\end{equation}
offering the potential for {\em relative} error control.  Unlike roundoff error however, the $\tt SpAMM$ error is deterministic,
leading to a non-associative algebra and the iterative development of error flows associated with the Lie bracket
\begin{equation}
\left[ \mat{a} , \mat{b} \right]_{\tau} = \mat{a} \ot \mat{b}-\mat{b} \ot \mat{a}  
=  \left[ \mat{a} , \mat{b} \right]
+ \mat{\Delta}^{a\cdot b}_{\tau} -\mat{\Delta}^{b\cdot a}_{\tau} \,.
\end{equation}

As in other $N$-body approximations \cite{}, the space filling curve (SFC) heuristic is employed here to preserve the locality 
of finite support functions \cite{}, an approach common for domain decomposition in computational physics \cite{}.
Also,  adaptive (persistence based) methods \cite{} and graph based algorithms \cite{} are available for developing locality in linear algebra.
Interestingly, locality enhancement is dimetric to recent distributed memory implementations of the $\tt SpGEMM$, which 
enjoy work load homogenisation through radomization, together with conventional communication strategies \cite{}.    
Differences between the $\tt SpAMM$ kernel and randomized approaches to the $\tt SpGEMM$ are reviewed in Ref.[\cite{}], and with 
the $\tt GEMM$ in Ref.\cite{}.
  
\section{Newton Shulz Iteration}

Here, we hope to understanding the flow of occlusion errors under Newton Schulz iteration, 
for ill-conditioned problems and with extreme permisive values of $\tau$. 


\subsection{Retaining the Eigenspace}
Gradients lack convergence properties
Iteration without orig drives away from basis
NS has both.  Difference between scalar iteration, Higham page 92.


\subsection{Idempotence }

\subsection{The Scaled Map}

\subsection{Alternative Formulations}
dual, stabilized and naive

\section{Occlusion Flows}

$\delta \mat{x}_k$ and $\delta \mat{z}_k$ arrize from itteration with $\ot$, and are deterministic 
flows away from the manifold of $\mat{s}$ determined by sensitivity of the NS iteration to these 
numerical insults. 


\begin{equation}
\delta \mat{x}^{\rm{naiv}}_k =   \delta  \widetilde{ \mat{z}}_{k} \cdot \mat{s} \cdot \widetilde{\mat{z}}_{k} 
                           +  \widetilde{\mat{z}}_{k} \cdot \mat{s} \cdot \! \delta \widetilde{\mat{z}}_{k} 
\end{equation}



\begin{equation}
\delta \mat{x}^{\rm{dual}}_k =   \delta  \widetilde{ \mat{y}}_{k} \cdot \widehat{\mat{z}}_{k} 
                           +  \widetilde{\mat{y}}_{k} \cdot \delta \widetilde{\mat{z}}_{k} 
\end{equation}


\begin{eqnarray}
\widetilde{\mat{x}}_k &=& f \left[\widetilde{\mat{z}}_{k-1} , \widetilde{\mat{x}}_{k-1} \right] \\ 
&=&
\tt{m} \left[ \widetilde{\mat{x}}_{k-1}\right] \cdot \widetilde{\mat{z}}^\dagger_{k-1}  
\cdot \mat{s} \cdot \widetilde{\mat{z}}_{k-1} \cdot \tt{m}\left[ \widetilde{\mat{x}}_{k-1} \right] 
\nonumber
\end{eqnarray}

\begin{equation}
\delta \mat{x}_k = {f}_{\delta \mat{z}_{k-1}}  \, \lVert \delta \mat{z}_{k-1} \rVert 
                              +  {f}_{\delta \mat{x}_{k-1}}   \, \lVert \delta \mat{x}_{k-1} \rVert 
                                                                                      + {\cal{O}} \left(  \tau^2 \right)
\end{equation}

generalized Gateaux differential

\begin{eqnarray}
f_{\delta \mat{z}_{k-1}} &=& \lim_{\tau \rightarrow 0} \frac{ f [ \mat{z}_{k-1} +\tau  \delta \widehat{\mat{z}}_{k-1}, \widetilde{\mat{x}}_{k-1} ]
-f [\, \mat{z}_{k-1}, \widetilde{\mat{x}}_{k-1} ]  }{\tau} \nonumber  \\[0.1cm] 
&=&{L}_{\widetilde{\mat{x}}_k}\left(\widetilde{\mat{z}}_{k} , \delta \widehat{\mat{z}}_{k-1} \right)  
\end{eqnarray}

\begin{eqnarray}
f_{\delta \mat{x}_{k-1}} &=& \lim_{\tau \rightarrow 0} \frac{ f [ \widetilde{\mat{z}}_{k-1}, \mat{x}_{k-1} + \tau \delta \widehat{\mat{x}}_{k-1} ]
-f [ \widetilde{\mat{z}}_{k-1}, \mat{x}_{k-1} ]  }{\tau} \nonumber  \\[0.1cm] 
&=&{L}_{\widetilde{\mat{x}}_k}\left(\widetilde{\mat{z}}_{k} , \delta \widehat{\mat{x}}_{k-1} \right)  
\end{eqnarray}

\begin{multline}
{L}_{\widetilde{\mat{x}}_k}\left(\widetilde{\mat{z}}_{k} , \delta \widehat{\mat{x}}_{k-1} \right) 
= \delta \widehat{\mat{x}}^\dagger_{k-1} \cdot   \tt{m}'\left[\mat{x}_{k-1} \right] \cdot 
\{\widetilde{\mat{z}}^\dagger_{k-1}  \cdot \mat{s} \cdot \widetilde{\mat{z}}_{k} \}  \\
+ \{ \widetilde{\mat{z}}^\dagger_{k} \cdot \mat{s} \cdot  \widetilde{\mat{z}}_{k-1} \} 
\cdot \tt{m}'\left[\mat{x}_{k-1} \right]  \cdot \delta \widehat{\mat{x}}_{k-1} 
\end{multline}

\begin{multline}
{L}_{\widetilde{\mat{x}}_k}\left(\widetilde{\mat{z}}_{k} , \delta \widehat{\mat{z}}_{k-1} \right) = 
\{ \tt{m}\left[\mat{x}_{k-1} \right]  \cdot \delta {\widehat{\mat{z}}}^\dagger_{k-1} 
 \cdot \mat{s} \} \cdot \widetilde{\mat{z}}_{k} \\
+\widetilde{\mat{z}}^\dagger_{k} \cdot \{ \mat{s} \cdot \delta {\widehat{\mat{z}}}_{k-1}
\cdot \tt{m}\left[\mat{x}_{k-1} \right]    \} 
\end{multline}

\begin{equation}
 \{ \widetilde{\mat{z}}^\dagger_{k} \cdot \mat{s} \cdot  \widetilde{\mat{z}}_{k-1} \} 
\rightarrow \mat{p}_+\left[\mat{s} \right]
\end{equation}

\begin{equation}
\{ \mat{s} \cdot \delta {\widehat{\mat{z}}}_{k-1}
\cdot \tt{m}\left[\mat{x}_{k-1} \right]    \} 
\rightarrow \mat{n}\left[\mat{s} \right]
\end{equation}

\section{Implementation}

\subsection{Methods}
FP, F08, OpenMP 4.0

\subsection{A Modified NS Map}

\subsection{$\delta \mat{x}_k$ and $\delta \mat{x}_k$ channels}
tau= Figure showing channels etc.  

\subsection{Convergence}
Map switching and etc based on TrX


\section{Ill-Conditioned Support}

\subsection{ 3,3 carbon nanotube with diffuse $sp$-function}
double exponential (Fig.)

\subsection{Water with triple zeta and double polarization}
Here's looking at you Jurg...

\section{Experiments}

\subsection{Occlusion Error Flows}
\begin{figure}[h]
  \caption{equation...}
 \includegraphics[width=3.5in]{8x_33_nanotube_cond10_tau-5.eps}
 \includegraphics[width=3.5in]{8x_33_nanotube_cond10_tau-3.eps}
\end{figure}
\begin{figure}[h]
  \caption{equation...}
 \includegraphics[width=3.5in]{8x_33_nanotube_cond10_compare_errors.eps}
\end{figure}

\subsection{Comments}

\begin{multline}
 \delta {\mat{z}}_{k-1} \approx \Delta^{\widetilde{\mat{z}}_{k-2} \cdot \tt{m}\left[ \widetilde{\mat{x}}_{k-2}\right]}_\tau 
+ \mat{z}_{k-2} \cdot \tt{m}'\left[ \widetilde{\mat{x}}_{k-2}\right] \cdot \delta \mat{x}_{k-2} \\
+\delta \mat{z}_{k-2} \cdot \tt{m} \left[\widetilde{\mat{x}}_{k-2} \right] 
\end{multline}

\begin{multline}
\lVert \delta {\mat{z}}_{k-1} \rVert \lesssim
\lVert \mat{z}_{k-2} \rVert \left( \;  \tau \, \lVert \tt{m} \left[\widetilde{\mat{x}}_{k-2} \right]  \rVert \right.   \\ \left.
+ \; \lVert \delta {\mat{x}}_{k-2} \rVert   \lVert \tt{m}' \left[\widetilde{\mat{x}}_{k-2} \right] \rVert \; \right)
\end{multline}

\begin{equation}
\lVert \mat{z}_{k} \rVert  \rightarrow \sqrt{\kappa\left(\mat{s} \right)}
\end{equation}

\subsection{Querry Surfaces}

\begin{figure}[h]
  \caption{equation...}
 \includegraphics[width=3.5in]{snapshot.png}
\end{figure}

\subsection{Comments}

\subsection{Found Contraction}

\subsection{Comments}
Pictures of the spamm structure

\section{Conclusion}

%%eg vs row-col picture.  Example of exact exchange w/DBSR 

\bibliography{MatrixFunctions}

\end{document}
