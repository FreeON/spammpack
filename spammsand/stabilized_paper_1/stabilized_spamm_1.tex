%% LyX 2.1.1 created this file.  For more info, see http://www.lyx.org/.
%% Do not edit unless you really know what you are doing.
%\documentclass[letterpaper,twocol,amsmath,aps,jcp,preprintnumbers,groupaddress,nofootinbib,tightenlines]{revtex4-1}
%\usepackage[latin9]{inputenc}
%\usepackage{textcomp}

\documentclass[letterpaper,twocolumn,amsmath,amsfont,amssymb,english,aps,jcp,preprintnumbers,groupaddress,nofootinbib,tightenlines]{revtex4}

%\documentclass[aps,prb,letterpaper,twocolumn,nofootinbib,showkeys]{revtex4-1}
%\documentclass[aps,amssymb,prl,letterpaper,twocolumn,nofootinbib,showkeys]{revtex4-1}

%\usepackage[backend=bibtex]{biblatex}






%    backend=biber,
%    style=authoryear,
%    natbib=true,
%    sortlocale=en_US,
%    url=false,
%    doi=true,f
%    eprint=false
%]{biblatex}
%\usepackage{hyperref}


\newcommand{\mat}[1]{\boldsymbol{#1}}

%%\hypersetup{pdftitle={FreeON Project Report 1}}
%\hypersetup{pdfauthor={Matt Challacombe and Nicolas Bock}}
%\hypersetup{pdfsubject={A SpAMM Stabilized Newton Schulz Preconditioner: Fighting Error with Error}}

%\bibstyle{aipnum4-1}

\begin{document}

\title{A SpAMM Stabilized Newton Schulz; Fighting Error with Error. }

% \author{Matt Challacombe}
% \email{matt.challacombe@freeon.org}
% \homepage{http://www.freeon.org}
% \affiliation{Theoretical Division, Los Alamos National Laboratory}

% \author{Nicolas Bock}
% \email{nicolasbock@freeon.org}
% \homepage{http://www.freeon.org}
% \affiliation{Theoretical Division, Los Alamos National Laboratory}

%\begin{abstract}
%Forward look
%\end{abstract}

\maketitle

\section{Eigen Basis Corruption}

\begin{equation}
\boldmath{\large \epsilon}_\perp = \left[ \tilde{\bm{x}}\, \cdot \, \tilde{\bm{x}}^\dagger 
                                         -\tilde{\bm{x}}^\dagger \,   \cdot \, \tilde{\bm{x}} \right]
                                     \end{equation}



\begin{equation}
\boldmath{\large \epsilon}_\perp = \left[ \tilde{\bm{x}},\tilde{\bm{x}}^\dagger  \right]
\end{equation}




\begin{eqnarray}
\widetilde{\mat{x}} &=& \mat{z} \, {\scriptstyle \otimes}_\tau  \left[  \mat{s} \,\,  {\scriptstyle \otimes}_\tau \, \mat{z}  \right] \\[0.1cm]
                    &=& \mat{z} \, {\scriptstyle \otimes}_\tau  \left[  \mat{s}\!\cdot \! \mat{z} + \mat{\delta}^{[sz]}  \right] \\[0.1cm]
                    &=& \mat{x} + \mat{\delta}^{z[sz]} + \mat{z} \cdot \mat{\delta}^{[sz]} %+ \mathcal{O}( \tau^2  ) 
 \end{eqnarray}


\end{document}
